\chapter{Introduction}
\label{chapter:introduction}

%\section{Overview}
%\label{section:overview}
The context of the rendering application always dictate the constraints applied to the methods and
algorithms used by the software for the image generation.
Modern approaches to realistic rendering could be roughly divided into distinct groups:
physically based rendering (light simulation approach) and plausible rendering (approximate
rendering, phenomenological approach). I will use the terms light transport simulation and simply,
rendering, interchangeably in this work.

Physically based rendering tries to simulate interaction of the light and matter taking into account
the laws of physics and our theoretical knowledge about the nature.

Plausible rendering, on the other hand, is intended to mimic optical effects observed by the human
eye or captured by camera, accounting the audience perception.

Monte Carlo methods of the light simulation are usually perceived as physically based methods.
Sometimes they are also used for relatively fast way to achieve desired plausible results regardless
of the physical correctness.

\section{Scope of the topic}
\label{section:scope}

Recent achievements in the approximations of reflective and refractive materials with Bidirectional
Reflectance Distribution Functions and decoupling it from the light integration made a big step in
the physically based visualization \cite{pharr2010physically}. Together with a modern hardware, it
is now possible to visualize a wide range of scenarios required by everyday tasks in engineering,
marketing and visual effects domains. The absence of the a big number of ad-hoc parameters allows to
start rendering without labour intensive manual preparations needed in the past for certain
algorithms \cite{DBLP:conf/mam/KettnerRSJK15}.

But the rendering of the subsurface scattered materials remains a bottleneck for most of the light
transport simulation scenarios. In this paper I study Monte Carlo rendering of the subsurface
scattering materials, both in the context of light simulation and approximate rendering.
The main interest is in the correctness of the methods and the consistency of the results between
them. With the discussion of the optimisation techniques applicable in certain scenarios.

I focus on the methods of the rendering of highly scattering isotropic materials which usually
appear as optically thick objects. These are examples of the materials which, in most cases, meet
this condition:
\begin{itemize}
    \item Non-organic
    \begin{itemize}
        \item Plastics, rubbers or waxes
        \item Marble (polished, rough, dusty)
        \item Snow
        \item Non photo real (cartoon and artistic stylisation)
    \end{itemize}
    \item Organics
    \begin{itemize}
    \item Optically thick liquids (milk, some kinds of juice)
        \item Skin (sweaty, dry, makeup)
        \item Food (cheese, meat, bread, some kinds of fruits)
    \end{itemize}
\end{itemize}
These are examples of materials out of the scope:
\begin{itemize}
    \item Highly transparent liquids (dirty/dusty water, ocean water, wine)
    \item Fluids (air, fog, dust)
\end{itemize}
I consider only isotropic media which are modeled by means of geometric
polygonal meshes. I do not consider spectral scattering. Only monochromatic light is simulated.
There are papers describing the efficient process of combined rendering of RGB spectrum with
corresponding multiple importance sampling weights.

There were many publications in the recent years related to rendering of the Subsurface Scattering
using a concept of the Photon Beam Diffusion \cite{Habel:2013:PBD:2600890.2600896}. This technique
demonstrates significant results and robustness especially while in combination with other
sampling approaches as UPBP \emph{Unifying Points, Beams, and Paths in Volumetric Light Transport
Simulation} \cite{krivanek14upbp}.

Despite the high potential of the above mentioned approach, there are two main reasons why it is not
described in my work.

First, it is based on a Bidirectional Path Tracing
\cite{Lafortune:1996:RPM:275458.275468}. Even though it is a powerful algorithm solving many
complicated phenomenon, it usually requires very complex implementation and still not that
ubiquitous among the available production rendering engines. 

The second complication is the requirement of the additional precaching step as it is required for
classic photon mapping approach or photon mapping for volumetric rendering \cite{Jensen:1998:ESL:280814.280925}.
There are several disadvantages of using light caching for rendering:
\begin{itemize}
    \item{Undesired preprocessing step is unfriendly to progressive (iterative) rendering}
    \item{Density of the generated point cloud (or other structure) limits the scope of the
    available parameters. I.e. restricts the effective width of the SSS by minimal distance between
    points such that point density $\approx$ mean free path length}
    \item{Photon mapping introduce some extra control parameters which often require the end user to
    learn the details of the underlying algorithms}
    \item{Additional memory requirements}
    \item{Workflow complications for storing additional data}
    \item{Possible flickering artifacts during animation}
\end{itemize}


\section{Volumetric path tracing with Next Event Estimation}
\label{section:udpt}
Although the classic Unidirectional Path Tracing \gls{UDPT} is proved to be a robust unbiased
estimator, it has some limitations.

A typical disadvantage of \gls{UDPT} is the high variance and subsequently strong noise component in
the rendered images of the scenes with highly non-uniform incoming illumination.
In practice, these scenes contain small but intensive emitters or direct sun light in the
image-based lighting scenes. The extreme worst case is the presence of the analytic light sources:
point lights with zero area or directional lights, having virtually infinite distance to the shaded
location.

Next Event Estimation (\gls{NEE}) is one of the common methods of reducing a variance of the path
tracing. The main idea of the \gls{NEE} is to estimate \textit{direct illumination} by sampling the
light source directly at each step of the path tracing. Then, the shadow test ray is needed to
ensure that the current light sampling position in visible by the surface.

\gls{NEE} can also be considered as the special case of the general Bidirectional Path Tracing
algorithm \gls{BDPT} \cite{Veach:94:BDPT}.
Using the vocabulary of the original paper, NEE algorithm is \textit{(m,2)-method} with the length
of light path equals 2 and length of the eye path $m\geq3$ is defined by the \gls{UDPT} settings.

The idea of direct light sampling can be applied to the volumetric path tracing as well.
The generalized theory of \gls{BDPT} in participating media was first described by Lafortune and
Willems in \cite{Lafortune:1996:RPM:275458.275468}.

Preforming the direct light estimation from inside the media with the refractive boundary proved to
be a non-trivial task. Due to the refraction on the boundary between materials with different index
of refraction (\gls{IOR}), the light from the emitter to the scattering point never travels by a
straight light.
Except the case of the normal incident rays. To satisfy the Fermat's principle \footnote{Fermat's
principle or the principle of least time states that the path taken between two points by a ray of
light is the path that can be traversed in the least time} we have to construct the path with the
middle point somewhere on the boundary.

There are no robust and fast enough methods for solving this complication in general way, to my
knowledge. Although, there are considerable iterative approaches described in the literature
last years \cite{holzschuch:hal-01083246}, \cite{10.1111:cgf.12681}, \cite{Koerner2016}.

In this work I have decided to consider the situation when the boundary refraction can be neglected.
In the other words, only the simplified case of materials with \gls{IOR}=1 is described.
This assumption leads to the approximation of the direct light path from inside the media to the
light source as a straight line. This certainly introduces an error in the simulation.
But the error is significant only for materials with low optical density. Light propagation in
highly scattering media, in contrast, is characterized by many scattering events. Which makes the
direction of the first ray less important for the future light simulation.

In practice, \gls{UDPT} almost always contains Russian Roulette optimization. Detailed description
is in section \ref{subsection:rr}. It proved to be a robust unbiased technique. But it may has some
non obvious numerical problems especially while being used in scenarios like volumetric rendering
(see chapter \ref{section:numerical})



\section{Intuition behind Diffusion approximation and BSSRDF formulation}

\subsection{Analogy to BRDF approximation}
\label{section:BSSRDF_intuition}
Formulate scattering in media in analogy to local surface reflectance (BRDF)

\subsection{Single- and Multiple scattering}
Decomposition into single- and multiscattering terms

\subsection{Searchlight problem formulation}
\label{section:searchlight}
Simplify model to the 'Searchlight Problem' to allow analytic solutions.
Perpendicular light beam to infinite surface \cite{Jacques1995}
\begin{figure}[h]
    \centering
    \includegraphics[width=\textwidth]{imgs/schemes/searchlight_disney}
    \caption{Schematic view of the searchlight experiment. To change.}
    \label{fig:searchlight_scheme}
\end{figure}


\section{Scattering in thin objects}
Highly diffusive Bidirectional Transmission Distribution Function (BTDF) can be used as a fast
approximation of subsurface scattering in thin materials. Many real world objects like sheets of
paper or foliage of the trees has strong scattering component. There is a simple and inexpensive
technique to approximate scattering in such objects by using the Lambertian BTDF. Apart from that,
there are other more correct methods described in the literature. The recent work on the rendering
of the paper material \cite{DBLP:journals/cgf/PapasMJ14} can be seen as the further advances of the
earlier research of representation of translucent materials \cite{Donner:2005:LDM:1186822.1073308}.


\ldots
