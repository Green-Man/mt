\section{Christensen-Burley profile importance sampling}
\label{section:burley_importance}

Imortance sampling of the aforementioned BSSRDF function is a vital part of the
rendering system used in real world production. Analogously to the original
paper derivation, the \gls{CDF} of the modified Christensen-Burley \gls{BSSRDF}
profile is expressed by the following expression: \comment{Todo: add full
derivation}
\begin{equation}\label{eq:burley_cdf}
CDF(R_d) = 1-\frac{1}{4}e^{-sx}-\frac{3}{4}e^{-\frac{1}{3}tx}
\end{equation}

The inversion of \eqref{eq:burley_cdf} Authors of original method suggests three
way of imp Sampling of each exponental separatly leads to six Importance Samples in total
in RGB mode.
Newton iterations show more robust results than sampling of the each exponent
separately.

Quote: Sampling can be difficult and expensive. Diffusion
integration requires distributing samples over the entire surface with known
density, preferably proportional to the diffusion profile. We can generate
such samples within a plane, but projecting this density through a solid can
oversample some parts of the surface and undersample others. Using MIS between
multiple planes \cite{King:2013:BIS:2504459.2504520}  helps prevent
undersampling but oversampling is still a problem. Each plane sample also
requires ray probing to find surface hits, and if there are multiple
intersections we build a conditional CDF based on the distance to each point,
then stochastically choose a point to sample. The per-sample cost is substantial