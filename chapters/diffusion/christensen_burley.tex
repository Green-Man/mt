\section{Christensen-Burley method}
The general problem discussed here is a search for a good diffusion
approximation model to fit Monte Carlo references by using analytic BSSRDF function.

The most recent research in this topic up to date
\cite{Christensen:2015:ARP:2775280.2792555}, \cite{Burley:disney_siggraph15}
suggested the \gls{BSSRDF} profile expressed as a combination of two exponential
functions with corresponding parametrization factor based on physical properties
of the media and proper normalization to ensure energy conservation. It was
proposed as a better function instead of Gaussian and Cubic profiles described
in the literature in the past and widely used.

The original normalized diffuse reflectance profile:
\begin{equation}\label{eq:burley_normalized}
R_d(r) = \dfrac{e^{-r/d}+e^{-r/3d}}{8\pi dr}
\end{equation}

Equation \eqref{eq:burley_normalized} represents normalized form.
with normalization over area and hemisphere ensures that no extra radiant energy
is emitted or absorbed\footnote{Absorption is not accouned during the
calculation of the shape of the diffusion profile. But is added late as
part of the albedo} during scattering:
\[
\int\limits_0^{2\pi}\int\limits_0^\infty R_d(r)rdrd\phi = 1
\]

Which leads to the form:
\begin{equation}\label{eq:burley}
R_{l=1}(r) = As\dfrac{e^{-sr}+e^{-sr/3}}{8\pi r}
\end{equation}
With:
\begin{itemize}
    \item{$r$ is a distance between the points of light, entering and leaving
    the surface (with the assumption of the planar geometry)}
    \item{$s$ is a function of $A$, empirically chosen by Christen and Burley
    to fit the Monte-Carlo references. For example, for the Searchlight
    scenario, authors suggest $s = 1.85 - A + 7|A-0.8|^3$}
    \item{ $A=\int_0^{\infty} R(r)2\pi rdr$ is the surface albedo (integral
    amount of the radiant exitance over whole area of the object relative to the
    overall irradiance). $A$ of the material depends on $\sigma_s$ and
    $\sigma_a$. The particular form of this function, empirically chosen by
    Christen and Burley to fit the Monte-Carlo references.}
\end{itemize}

This model includes single scattering, so no expensive separate
calculation is needed.

Cons of the model:
\begin{itemize}
  \item {As all other Diffusion approximation models, still follows the infinite
  plane geometry assumption. So fails on large distance parameters. }
  \item {Surface normals are not considered. Thay is why cavities are probles
  (without special treatment) and bump details are blurried}
  \item{Not that trivial importance sampling. See details in section
  \ref{section:burley_importance}}
\end{itemize}