\section{Scope of the topic}
\label{section:scope}
In this paper I study Monte Carlo rendering of the subsurface scattering
materials, both in the context of light simulation and approximate rendering.
The main interest is in the correctness of the methods and the
consistency of the results between them. With the discussion of the optimisation
techniques applicable in certain scenarios.

I focus on the methods of the rendering of highly scattering isotropic materials
which usually appear as optically thick objects.

These are examples of the materials which, in most cases, meet this condition.
\begin{itemize}
    \item Non-organic
    \begin{itemize}
        \item Marble (polished, rough, dusty)
        \item Plastics or waxes
        \item Snow
        \item Thin sheets (paper, foliage, cloth)
        \item Non photo real (Cartoon style and stylisation)
    \end{itemize}
    \item Organics
    \begin{itemize}
    \item Optically thick liquids (milk, some kinds of juice)
        \item Skin (sweaty, dry, makeup)
        \item Food (cheese, meat, bread, some kinds of fruits)
    \end{itemize}
\end{itemize}
These are examples of materials out of the scope:
\begin{itemize}
    \item Highly transparent liquids (dirty/dusty water, ocean water, wine)
    \item Fluids (air, fog, dust)
\end{itemize}
I consider only isotropic media which are modeled by means of geometric
polygonal meshes. The constraint of the closed shape in not applicable to all
the methods discussed and will be discussed later.

I do not consider spectral scattering. Only monochromatic light is simulated.
There are papers describing the process of combined rendering of RGB with
corresponding importance sampling. To add.

Why not to use cached data structures approach:
\begin{itemize}
    \item{Ease of use for the user. Don't want to introduce a lot of control
    parameters which often require the end user to learn the details of the
    underlying algorithms}
    \item{Undesired preprocessing step. Unfriendly to progressive rendering}
    \item{Density of the generated point cloud (or other stucture) limits the
    scope of the available parameters. It restrics the effective width of the
    SSS by minimal distanse between points, for example. Point density $\approx$
    mean free path}
    \item{Possible flickering artifacts during animation}
    \item{Additional memory requirements}
    \item{Workflow complications for storing additional data}
\end{itemize}

Other things to consiger while choosing method:
\begin{itemize}
  \item Lighting restrictions \emph{Analytic light} sources and \emph{area lights and IBL} friendly
  \item Parametrization: Artist friendly or physically correct
  \item Homogeneous or inhomogeneous
  \item Performance. Is it suitable for real time application?
\end{itemize}