\section{Monte-Carlo integration with importance sampling}
\label{section:importance_sampling}
According to the theory of the light transport, the behaviour of the light can be expressed by the
equation of radiative transfer in homogeneous media \cite{Lafortune:1996:RPM:275458.275468}:
\begin{equation}
\label{rte}
L_o(\vec{x_0}, \Theta) = \int_{A} \int_{\Omega} f(\vec{x}_i-\vec{x}_o,\omega_o,\omega_i)
L_i(\vec{x}_i-\vec{x}_o, \omega_o, \omega_i) d\omega_i e^{-ks} + L_e(\vec{x}_i-\vec{x}_o, \omega_o)
e^{-kd}
\end{equation}

In this chapter we can proceed without detailed description of every part of the equation \ref{rte}.
The main idea at this point is that computing of the relation between incoming and outgoing radiance
requires double integration of complicated and almost arbitrary functions and, in general, has no
analytical solution. This integral formulation is a good candidate for evaluation using Monte-Carlo
technique.

To evaluate the arbitrary integral of the form:
\[
I = \int_\Theta f(x) dx
\]
we have to evaluate the integrand function at the independent randomly sampled points $x_i$ such
that $f_i = f(x_i)$. In case if the probability density function of the sampling is $p(x_i)$ and $N$
number of samples is taken, the estimated value of the integral is \cite{hammersley64}:
\begin{equation}
\label{mc_estimator}
\hat{I}=\frac{1}{N}\sum_{i=1}^N\frac{f(x_i)}{p(x_i)}
\end{equation}

In special case of uniformly sampled function the $p(x_i)=1$ and the Monte Carlo estimator
\ref{mc_estimator} has a form of quadrature rule where the sampling points are chosen randomly
instead of uniform grid:
\begin{equation}
\label{uniform_estimator}
\hat{I}=\frac{1}{N}\sum_{i=1}^N f(x_i)
\end{equation}

The Monte Carlo integration is the main method of estimating the light transport integrals due to
it's robustness regardless of the number of dimensions and the form of the integrand
\cite{Veach:1998:RMC:927297}

The uniform Monte Carlo estimator is often modified to reduce variance due to the slow $O(\sqrt{N})$
convergence. It is desirable to find the sampling distribution $p(x_i)$ such that it has as close a
possible form to the resulting integral $I$. This technique is called \emph{importance sampling}. In
path tracing context it requites the sampling of the paths which carry the most of the light
contribution to the resulting image. The problem of finding the right sampling for arbitrary
probability density function can be a complicated task in general. One practical problem of this
kind and possible solutions of it are discussed in the \ref{section:burley_importance} section of
this thesis.