%\clearemptydoublepage
\newpage

\phantomsection
\addcontentsline{toc}{chapter}{Outline of the Thesis}

\begin{center}
	\huge{Outline of the Thesis}
\end{center}

%--------------------------------------------------------------------
\section*{Part I: Introduction and Theory}

\textbf{ Chapter 1: Introduction}\\
The first chapter defines the the scope of the work. The high level overview of the Monte Carlo
methods which are used during the study and implemented.
The description of the light simulation in the participating media and possible variance reduction
techniques takes place from the general point of view without mathematical formalism. An overview of
the advantages and limitations of the each of the method for practical use is also given in this
chapter.

\textbf{Chapter 2: Theory}\\
Chapter 2 contains theoretical summary of the Monte Carlo methods with attention on such concepts,
which are important in the context of this work, such as light integration, importance sampling and
multiple importance sampling.
Theory of the light transport in participating media with emission, attenuation and scattering laws.
The theoretical introduction into the diffusion theory and diffusion approximation approach of light
integration in participating media.

%--------------------------------------------------------------------
\section*{Part II: Implementation and results}

\textbf{ Chapter 3: Path tracing integrator and Next Event Estimation}\\
The path tracing implementation details are discussed here with additional Direct Light Sampling
technique. Correctness of the method, and problems in practical usage such as numerical errors.
Comparison of the performance between two methods and the combination of them using MIS.

\textbf{ Chapter 4: Diffusion Approximation}\\
The practical details of the implementation of the Diffusion Approximation light integrator.
Sampling strategies and precomputed BSSRDFs. Special case of the Christenen-Burley BSSRDF profile
and fitting to Monte-Carlo references. Importance sampling of the precomputed profiles and and
parametrization issues.

Special section of this chapter is dedicated to the study of the limitations of the Diffusion
approximation from practical point of view.
