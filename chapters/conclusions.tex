\chapter{Conclusions and summary}
This master thesis describes several aspects of the light simulation in participating media from
practical point of view. The main focus is on the applications which require the rendering of the
dense, isotropic media without refractive boundary.

Unbiased Monte Carlo methods have to be employed in cases where the correctness of the rendered
image is key factor. We showed that the classic unidirectional volumetric path tracing can be
enhanced by the addition of the Next Event Estimation technique in participating media.
This PTDL sampling method proved to be beneficial in certain lighting scenarios. To be used in
robust production rendering solutions it has to be combined with classical brute force path tracing
estimator (section \ref{section:ptdl_implement}).

The problems of numerical cancellations of the path tracing estimators could alter the expected
results in the special cases of highly scattered media using Russian Roulette.
The proposed empirical technique gives an estimation on the limits of the path termination
probability parameter to avoid energy loss due to numerical errors (section
\ref{section:numerical}).

In the cases when the correctness of the simulation is not the main goal, the approximations to the
unbiased methods can be used to achieve better convergence time. The Diffusion Approximation is one
the most popular methods used in the industry for rendering SSS materials. One of the recent
researches in this field by Brent Burley and Per Christensen proposed in 2015 the efficient BSSRDF
profile as precomputed solution. In our work we studied the original profile by comparison of the
approximated solutions with the references. The alternative form of this profile is proposed
(section \ref{section:cb_modification}) with suitable parameterization
(\ref{section:cb_parametrization}). Two methods of the importance sampling of this BSSRDF were also
implemented and studied. The numerical root finding showed to be the preferable way of CDF inversion
(section \ref{section:burley_importance}).

According to the approximative nature of the precomputed diffusion method of SSS rendering, the
certain application domain have to be defined. Only the materials within limited range of properties
can be rendered relatively correctly. To find these limits we conducted a number of experiments
comparing the simulation results of Diffusion approximation with Monte Carlo references. Section
\ref{section:error_measure} describe the acceptability criteria. And the detailed description of the
results are provided in \ref{section:measurements_results} together with some real world materials
measurements data.
