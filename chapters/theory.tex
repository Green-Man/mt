\chapter{Theory}
\label{chapter:theory}

\section{Monte Carlo methods}
\label{Monte-Carlo}
\cite{Lafortune:1996:RPM:275458.275468} has short but decent description of the integrals on the
surface and in the media to integrate using with Monte Carlo approach.
A nice picture explaining BDPT is also there.

\gls{MIS} overview.
\subsection{Importance sampling}
\label{ImportanceSampling}
\emph{In order to reduce variance,  it is desirable to importance sample reflected rays by
preferentially considering paths carrying high energy.} \emph{The benefit of stratified importance
sampling  within  the  context  of  physically-based rendering  has  certainly  been  justified by 
the  work  of Cook  [1986].   Since  Shirley demonstrated  how  to  efficiently  sample  the 
traditional  Phong BRDF  [Shirley  1990]  and Lafortune  introduced  a  generalization of  this 
cosine-lobe  model  [Lafortune  et  al.  1997], a  reasonable approach to importance sampling an
arbitrary BRDF has been to sample  a  best-fit approximation  of  one  of  these  simpler  models.
Although this technique marks a clear improvement over random sampling,  it has several drawbacks.
First, it is not always trivial to approximate the many complex BRDFs that exist in nature with one
of these models.}

\subsection{Russian roulette}
\label{subsection:rr}
Not a variance reduction in mathematical sense. Strictly speaking, it it even
increases the variance of the estimator. But reduces to cost of sampling in
average \cite{Veach:1998:RMC:927297}, \cite{Csi03variancereduction}

Russian roulette is an indispensable technique in transport calculations, since it allows otherwise
infinite random walks to be terminated without bias \cite{Veach:1998:RMC:927297}

\ldots

\section{Light simulation in participating media}
\label{section:light_simulation_theory}
\subsection{Emission}
\subsection{Absorption}
\subsection{Scattering}

\subsection{Henyey-Greenstein}
\label{section:phasefunction}
\cite{Gkioulekas:2013:IVR:2508363.2508377} made a great user study.
Take home notes from it: \gls{hg} is not general enough (can't represent some
materials). Uniform parametrization needs requires to expose $g^2$ to the user.

\emph{TODO: while inroducing phase function, try to render single photon inside
the media and visualize it with plotly as Andreas did in pseudo 3d. To illustrate
random walk directions with different Henyey-Greenstein \textbf{g}.}

\section{The Diffusion Approximation}
Applies to the highly scattering media: relation to the Central Limit Theorem.
The underlying approach is similar to using a normal distribution for the sum of
random variables.
In the latter setting, the central limit theorem permits one to replace the analytically intractable sum of random variables by
an appropriately chosen normal random variable.

Probably first description is \cite{Stam1995}
Main paper: A Practical Model for Subsurface Light Transport \cite{Jensen:2001:PMS:383259.383319}

Geometric assumption on the surface as a flat, semi-infinite slab at (any
point). This is reasonable only if scattering distance is small with respect to
surface curvature and thickness. \textbf{Multiple scattering} diffusion
component is approximately determined. The \textbf{single scattering} is
separately integrated.
