\chapter{Conclusions and summary}
The practical aspects of light simulation in participating media were studied during the work on
this Master thesis from the practical point of view. We assumed, that a large number of applications
fall into the limited scope of materials, described as dense, isotropic media without refractive
boundary.

As the correctness of the rendered image can be a key factor in certain applications, the unbiased
Monte Carlo method have to be employed. We showed that the classic unidirectional volumetric path
tracing can be enhanced by the adding Next Event Estimation in participating media. This PTDL
sampling method proved to be beneficial in certain scenarios and have to be combined with brute
force path tracing to be used in production rendering solutions (section
\ref{section:ptdl_implement}).

The problems of numerical cancellations of path tracing estimators could alter the expected
estimation in the special cases of highly scattered media using Russian Roulette.
The proposed empirical technique gives an estimation on the limits of the path termination
probability parameter to avoid energy loss due to numerical errors (section
\ref{section:numerical}).

In the cases when the correctness of the simulation is not the main goal, the approximations to the
unbiased methods can be used to achieve better convergence time. The Diffusion Approximation is one
the most popular methods used in the industry for rendering SSS materials. One of the recent
researches in this field by Brent Burley and Per Christensen proposed the efficient BSSRDF profile
as precomputed solution. In our work we studied the proposed profile by comparison of the
approximated solutions with the references and proposed alternative form of this profile (section
\ref{section:cb_modification}) and parametrization (\ref{section:cb_parametrization}). Two methods
of importance sampling of this BSSRDF were also implemented and studied. The numerical root finding
showed to be the preferable way of CDF inversion (section \ref{section:burley_importance}).

According to the approximative nature of the precomputed diffusion method of SSS rendering, the
certain application domain have to be defined. Only the materials within limited range of properties
can be rendered relatively correctly. To find these limits we conducted a number of experiments
comparing the simulation results of Diffusion approximation with Monte Carlo references. Section
\ref{section:error_measure} describe the acceptability criteria. And the detailed description of the
results are provided in \ref{section:measurements_results} together with some real world materials
measurements data.
