\chapter{Diffusion Approximation}
\label{chapter:da}

\section{Parametrization}
Artists prefer to define surface albedo and mean multiple scattering distance
(characteristic or visible depth of the SSS). The radiative transport scattering
coefficients must therefore be inferred by inverting the diffusion equation.
Test: as seen at the figure \ref{fig:dragon_ptdl} at page \pageref{fig:dragon_ptdl}

The scattering was parameterized by the volume scattering and absorption
coefficients $\sigma_s$ and $\sigma_a$ (or, equivalently, the volume
scattering albedo $\alpha = \sigma_s/\sigma_t = \sigma_s/(\sigma_s + \sigma_a)$
and volume mean free path length $l = 1/\sigma_t = 1/(\sigma_s + \sigma_a)$.)
In follow-up work, Jensen and Buhler \cite{Jensen:2002:RHR:566570.566619}
introduced a more intuitive parameterization of subsurface scattering:
surface albedo (diffuse surface reflectance) $A = \int\limits_0^\infty
R(r)2\pi r dr$ and diffuse mean free path length $l_d$ on the surface.

Such "albedo inversion" must typically done with iterative or tabulated methods,
both of which can have trouble with high albedos. Another challenge is that
diffusion profiles are often too blurry given that they typically only model
multiple scattering, leading to the use of two or more profiles combined in an
ad hoc manner.


\section{Christensen-Burley method}
The general problem discussed here is a search for a good diffusion
approximation model to fit Monte Carlo references by using analytic BSSRDF function.

The most recent research in this topic up to date
\cite{Christensen:2015:ARP:2775280.2792555}, \cite{Burley:disney_siggraph15}
suggested the \gls{BSSRDF} profile expressed as a combination of two exponential
functions with corresponding parametrization factor based on physical properties
of the media and proper normalization to ensure energy conservation. It was
proposed as a better function instead of Gaussian and Cubic profiles described
in the literature in the past and widely used.

The original normalized diffuse reflectance profile:
\begin{equation}\label{eq:burley_normalized}
R_d(r) = \dfrac{e^{-r/d}+e^{-r/3d}}{8\pi dr}
\end{equation}

Equation \eqref{eq:burley_normalized} represents normalized form.
with normalization over area and hemisphere ensures that no extra radiant energy
is emitted or absorbed\footnote{Absorption is not accouned during the
calculation of the shape of the diffusion profile. But is added late as
part of the albedo} during scattering:
\[
\int\limits_0^{2\pi}\int\limits_0^\infty R_d(r)rdrd\phi = 1
\]

Which leads to the form:
\begin{equation}\label{eq:burley}
R_{l=1}(r) = As\dfrac{e^{-sr}+e^{-sr/3}}{8\pi r}
\end{equation}
With:
\begin{itemize}
    \item{$r$ is a distance between the points of light, entering and leaving
    the surface (with the assumption of the planar geometry)}
    \item{$s$ is a function of $A$, empirically chosen by Christen and Burley
    to fit the Monte-Carlo references. For example, for the Searchlight
    scenario, authors suggest $s = 1.85 - A + 7|A-0.8|^3$}
    \item{ $A=\int_0^{\infty} R(r)2\pi rdr$ is the surface albedo (integral
    amount of the radiant exitance over whole area of the object relative to the
    overall irradiance). $A$ of the material depends on $\sigma_s$ and
    $\sigma_a$. The particular form of this function, empirically chosen by
    Christen and Burley to fit the Monte-Carlo references.}
\end{itemize}

This model includes single scattering, so no expensive separate
calculation is needed.

Cons of the model:
\begin{itemize}
  \item {As all other Diffusion approximation models, still follows the infinite
  plane geometry assumption. So fails on large distance parameters. }
  \item {Surface normals are not considered. Thay is why cavities are probles
  (without special treatment) and bump details are blurried}
  \item{Not that trivial importance sampling. See details in section
  \ref{section:burley_importance}}
\end{itemize}

\section{Change in the Christensen-Burley profile to better fit Monte
Carlo references}


\textit{Note about Monte Carlo references:
There are consistent results achieved for Searchlight experiment
(semi-infinite slab with narrow pencil beam of light pointing from top in the normal
direction). In which brute force Monte-Carlo simulation in P2 (path tracing) was
compared to PTDL approach and also to hybrid technique used in Mitsuba
renderer. All three methods produce the same result in the scope of their
capabilities. It give us some assurance of the correctness of the Monte-Carlo
algorithms which are used as the references.}


\begin{figure}
    \centering
    \begin{subfigure}{0.48\textwidth}
        \includegraphics[width=\textwidth]{imgs/plots/burley_old_fitting}
        \caption{Original Christensen-Burley}
    \end{subfigure}
    \quad
    \begin{subfigure}{0.48\textwidth}
        \includegraphics[width=\textwidth]{imgs/plots/burley_new_fitting}
        \caption{Proposed}
    \end{subfigure}

    \caption{Comarison between the original and proposed techniques in log-y
    scale. Monte Carlo references are inthin lines. TODO: make clear plots and
    accompany them with rendered images}
    \label{fig:burley_fitting}
\end{figure}

As we can see from the graph \ref{fig:burley_fitting}, there is a tendency for
the original model to overestimate outgoing radiant exitance in the region close
to the light source (around $x\leq=3$). This trend was observed in many
experiments with different optical properties of the material.

The proposed model has different the factorization of the exponents than the
original \gls{BSSRDF} profile. Instead of using unified parameter $s$, I propose
to split it into two factors for each of the exponent:

\begin{equation}\label{eq:burley_modified}
R_{l=1}(r) = A\dfrac{se^{-sr}+te^{-tr/3}}{8\pi r}
\end{equation}

It still allows to easily normalize the model and use the same technique for
importance sampling. The particular form of s and t is based on the original
function developed by Christensen et al., but has a linear function coefficient
to get better results.

\ldots

\section{Christensen-Burley profile importance sampling}
\label{section:burley_importance}

Importance sampling of the BSSRDF function is a vital part of the
rendering system used in practice. The short theory of importance sampling is given in the section
\ref{ImportanceSampling}. According to it, we need to derive \gls{CDF} of the profile and invert it to find
the corresponding sampling. Analogously to the derivation provided in the original paper
\cite{Christensen:2015:ARP:2775280.2792555}, the \gls{CDF} of the modified Christensen-Burley
\gls{BSSRDF} profile \ref{eq:burley_modified} is expressed by the following expression:
\[
\label{eq:burley_cdf}
CDF(R_d(r)) = \int_{0}^{r} R_d(x)2\pi xdx = \int_{0}^{r} A\frac{s e^{-sx}+t e^{-tx/3}}{8\pi x}2\pi
xdx = A(1-\frac{1}{4}e^{-sr}-\frac{3}{4}e^{-\frac{tr}{3}})
\]

Let $\xi$ be uniformly distributed random variable withing the range from 0 to 1. To find the
sampling of the radius $r$ according to the \ref{eq:burley_modified}, the following equation have to
be solved (inversion step):
\begin{equation}
\label{eq:inversion}
\xi=A(1-\frac{1}{4}e^{-sr}-\frac{3}{4}e^{-\frac{tr}{3}})
\end{equation}

This step would in theory result in the corresponding function $r(\xi)$ which can be used to
generate samples. But the fact that there is no analytical solution to equation
\ref{eq:inversion} makes it not that trivial to use in practice.
There are three main approaches to overcome this difficulty:
\begin{enumerate}
  \item Precompute the solution before rendering and use a lookup table with interpolation
  during the runtime
  \item Sample each of the exponential of the right side of the equation \ref{eq:burley_modified}
  separately and combine the result
  \item Solve the equation numerically in runtime for each sample on demand
\end{enumerate}

The derivative of CDF:
\[
\frac{dCDF(R_d(r))}{dr} = \frac{A}{4}(se^{-sr}+te^{-\frac{tr}{3}})
\]

Sampling each of the exponential separately leads to six importance Samples in
total in RGB mode.

\textbf{The main point of this section:}
Couple of Newton iterations to invert CFD show more robust results than
combination of the sampling of the each exponent separately.

\emph{Quote: Sampling can be difficult and expensive. Diffusion
integration requires distributing samples over the entire surface with known
density, preferably proportional to the diffusion profile. We can generate
such samples within a plane, but projecting this density through a solid may
lead to oversampling of some parts of the surface and undersampling of the
others.
Using MIS between multiple planes \cite{King:2013:BIS:2504459.2504520}  helps prevent
undersampling but oversampling is still a problem. Each plane sample also
requires ray probing to find surface hits, and if there are multiple
intersections we build a conditional CDF based on the distance to each point,
then stochastically choose a point to sample.}

\section{BSSRDF parametrization}
Physically based parameters $\sigma_s$ and $\sigma_a$ are used for the
description of the material appearence in the context of subsurface
scattering. The \textit{volumetric scattering albedo} coefficient
$alpha = \sigma_s / \sigma_s+\sigma_a$ is used for the same purpose while using
diffusion approximation


In contrast, BSSRDF profile for the diffusion approximation proposed by Burley
and Christensen relays on \textit{surface albedo} (diffuse surface reflectance)
$A$ of the material:



It can not be easily analytically computed at runtime out of $\sigma_s$ and $\sigma_a$. But we need it for unified parametrization on the materials independent of the rendering technique.

By rendering different materials with various oprical properties and using curve
fitting I've come up to the following expression:
\[
A(\alpha) = a + b(1- e^{-c\alpha})
\]

with parameters $a = 0.0699, b = -3.97\cdot 10^{-05}, c =-10.0$

The results from the following plot show good match for Burley profiles (thick
lines) to Monte Carlo references (thin lines) for high albedo materials (top
four curves). But there is a significant difference for low albedo materials
with approximately $A<0.3$

Note that two lower thick lines does not not correspond to lower two thin lines.
And this result is consistent with the assumptions of the diffusion theory of
the light transport in participating media.



\section{Sampling methods}

\begin{itemize}
    \item{Infinite plane disk sampling \cite{Jensen:2001:PMS:383259.383319}}
    \item{Multiresolution radiosity caching \cite{Christensen:2012:MRC:2343045.2343108}}
    \item{Bidirectional Lightcuts \cite{Walter:2012:BL:2185520.2185555}}
    \item{Improved disk sampling \cite{King:2013:BIS:2504459.2504520}}\\
    $\star$ \textit{Editors' choise}. This method was used in my implementation.
    TODO: Detailed descriprion.
\end{itemize}


