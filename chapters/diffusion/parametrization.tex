\section{BSSRDF parametrization}

Artists prefer to define surface albedo and mean multiple scattering distance
(characteristic or visible depth of the SSS). The radiative transport scattering
coefficients must therefore be inferred by inverting the diffusion equation.


The scattering was parameterized by the volume scattering and absorption
coefficients $\sigma_s$ and $\sigma_a$ (or, equivalently, the volume
scattering albedo $\alpha = \sigma_s/\sigma_t = \sigma_s/(\sigma_s + \sigma_a)$
and volume mean free path length $l = 1/\sigma_t = 1/(\sigma_s + \sigma_a)$.)
In follow-up work, Jensen and Buhler \cite{Jensen:2002:RHR:566570.566619}
introduced a more intuitive parameterization of subsurface scattering:
surface albedo (diffuse surface reflectance) $A = \int\limits_0^\infty
R(r)2\pi r dr$ and diffuse mean free path length $l_d$ on the surface.

Such "albedo inversion" must typically done with iterative or tabulated methods,
both of which can have trouble with high albedos. Another challenge is that
diffusion profiles are often too blurry given that they typically only model
multiple scattering, leading to the use of two or more profiles combined in an
ad hoc manner.

Physically based parameters $\sigma_s$ and $\sigma_a$ are used for the
description of the material appearence in the context of subsurface
scattering. The \textit{volumetric scattering albedo} coefficient
$alpha = \sigma_s / \sigma_s+\sigma_a$ is used for the same purpose while using
diffusion approximation.


In contrast, BSSRDF profile for the diffusion approximation proposed by Burley
and Christensen relays on \textit{surface albedo} (diffuse surface reflectance)
$A=\int_0^{\infty} R(r)2\pi rdr$ of the material.

There is are way of approximation  analytically computed at runtime out of
$\sigma_s$ and $\sigma_a$. 

By rendering different materials with various oprical properties and using curve
fitting I've come up to the following expression:
\[
A(\alpha) = a + b(1- e^{-c\alpha})
\]

with parameters $a = 0.0699, b = -3.97\cdot 10^{-05}, c =-10.0$

The results from the following plot show good match for Burley profiles (thick
lines) to Monte Carlo references (thin lines) for high albedo materials (top
four curves). But there is a significant difference for low albedo materials
with approximately $A<0.3$

Note that two lower thick lines does not not correspond to lower two thin lines.
And this result is consistent with the assumptions of the diffusion theory of
the light transport in participating media.