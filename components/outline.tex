%\clearemptydoublepage
\newpage

\phantomsection
\addcontentsline{toc}{chapter}{Outline of the Thesis}

\begin{center}
	\huge{Outline of the Thesis}
\end{center}

%--------------------------------------------------------------------
\section*{Part I: Introduction and Theory}

\textbf{ Chapter 1: Introduction}\\
This chapter defines the the scope of the work and high level
overview of the Monte Carlo methods discussed and implemented. The description
of the light simulation in the participting media and possible variance reduction techniques
takes place from the general point of view. An overview
of the limitations of the each of the method in also given in this chapter.

\textbf{Chapter 2: Theory}\\
Chapter 2 contains theorical summary of the Monte Carlo methods with attention
on such concepts, which are important in the context of this work, such
as light integration, importance sampling and multiple importance sampling.
Theory of the in ligth transport in participating media with emission,
attenuation and scattering laws.
The parametrization of the optical properties of the media for physically based
simulations and approximations. The intoduction into the diffusion theory.

%--------------------------------------------------------------------
\section*{Part II: Implementation and measurements}

\textbf{ Chapter 3: Pathtracing integrator and Next Event Estimation}\\
Here the pathtracing implementation details are discussed with additional Direct
Light Sampling technique. Correctness of the method and problems in practical
usage. Comparison of the performance between two methods and the combination od
them using MIS.

\textbf{ Chapter 4: Diffusion Approximation}\\
The practical details of the implementation of the Diffusion Approximation
light integrator. Sampling starategies and precomputed BSSRDFs. Special case of
the Christenen Burley BSSRDF profile and fitting to Monte-Carlo references.
Importance sampling of the precomputed profiles and and parametrization issue.
\newline
\newline
\comment{Currently in TODO status}
\textbf{ Chapter 5: Comparisons and measurements}\\
Measurements, data and comparisons between different
implementations. Convergence graphs.
