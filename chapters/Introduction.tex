\chapter{Introduction}
\label{chapter:Introduction}

\section{Intro}

\subsection{Real world meterial approach}
\begin{itemize}
    \item Organics
    \begin{itemize}
      \item Skin (sweaty, dry, makeup)
      \item Fruits (layered, uniform, non-uniform)
      \item Food (Cheese, meat, bread)
    \end{itemize}
    \item Non-organic
    \begin{itemize}
      \item Marble (polished, rough, dusty)
      \item Plastics (wax, etc othe non-organic uniform solid bodies)
    \end{itemize}
    \item Liquids (milk, wine, juice, dirty/dusty water, ocean water)
    \item Fluids (air, fog, dust)
    \item Snow
    \item Thin sheets (paper, foilage, cloth)
    \item Non photoreal (Cartoon style and stylisation)
\end{itemize}

\subsection{Things to consiger while choosing method}
\begin{description}
  \item [Lighting restrictions] \emph{Analytic light} sources and \emph{area lights and IBL} friendly
  \item [Multiple object friendly]
  \item [Preprocess] Is additional preprocess step needed (point cloud generation)
  \item [Parametrization] Artist friendly or physically correct
  \item [Performance] Is it a real time approximation for OpenGL
  \item [Flickering] How suitable for animation (does it introduce additional flickering)
  \item [Motion blur] How well it works with motion blur (SSS and volumetrics)
  \item [Closed objecs only] Does it supports single sided geometry
  \item Homogeneous or inhomogeneous
  \item Istropic or anisoisotropic media
\end{description}

Light integration

Monte Carlo method

Path tracing

Unidirectional path tracing

Volumetric Path Tracing: Path tracing with light scattering inside the media

Next Event Estimation

\subsection{Unidirectional path tracing}
high variance and subsecuently strong noise component in the image

\subsection{Unidirectional path tracing with direct light sampling}
Next Event Estimation (\gls{nee}) is one of the common methods of reducing a
variance of the path tracing algorithm is. The main idea is to estimate
\textit{direct illumination} by sampling the light source directly at each step
of the path tracing. Then, the shadow test ray is needed to ensure that the
current light sampling position in visible by the surface.

\gls{nee} can also be considered as the corner case of the general
Bidirectional Path Tracing algorithm \cite{Veach:94:BDPT} with the length of light
path (TODO: check the params) m=2 and length of the eye path is defined by the
unidirectional path tracing.

Preforming the direct light estimation from inside the media with the refractive
boundry proved to be a non-trivial task. Due to the refraction of the
light on the boundary, the path from the scattering point to the light source is
never a straight light. It was impossible to find any fast enough and
robust method for solving this complication in general way. Although, there
are considerable iterative approaches described in the literature  during last
years \cite{holzschuch:hal-01083246}, \cite{10.1111:cgf.12681}, \cite{Koerner2016}.


