\section{Christensen-Burley method}
The general problem discussed here is a search for a good diffusion approximation model to fit Monte
Carlo references by using analytic BSSRDF function.

The most recent research in this topic up to date
\cite{Christensen:2015:ARP:2775280.2792555}, \cite{Burley:disney_siggraph15}
suggested the \gls{BSSRDF} profile expressed as a combination of two exponential functions with
corresponding parametrization factor based on physical properties of the media and proper
normalization to ensure energy conservation. It was proposed as function which better fit Monte
Carlo references than widely used Gaussian and Cubic profiles.

The original proposed diffuse reflectance profile:
\begin{equation}\label{eq:burley}
R_{d,l=1}(r) = As\dfrac{e^{-sr}+e^{-sr/3}}{8\pi r}
\end{equation}
With:
\begin{itemize}
    \item{$r$ is a distance between the points of light, entering and leaving
    the surface (with the assumption of the planar geometry)}
    \item{$s$ is a function of $A$, empirically chosen by Christen and Burley
    to fit the Monte-Carlo references. For example, for the Searchlight
    scenario, authors suggest $s = 1.85 - A + 7|A-0.8|^3$}
    \item{ $A=\int_0^{\infty} R(r)2\pi rdr$ is the surface albedo (integral
    amount of the radiant exitance over whole area of the object relative to the
    overall irradiance). Surface albedo of the material depends on $\sigma_s$
    and $\sigma_a$. The particular form of this function, empirically chosen by
    Christen and Burley to fit the Monte-Carlo references.}
\end{itemize}


Note, that here and in further calculations the formulas for the diffusion reflectance will be
provided with the assumption of the unit mean free path $l = 1/(\sigma_s + \sigma_a)=1$.
\comment{To explain the why}
This does not brake the principal shape of the profile. And the implementation of the arbitrary
spacial scaling does not expose any difficulties.

In contrast to the other diffusion approximation techniques proposed in the past (Dipole, Quantized
diffusion), this model already includes single scattering component, so no expensive separate
calculation is needed.


According to the authors, the main disadvantages of the model are:
\begin{itemize}
  \item {As all other Diffusion approximation models, it still follows the
  infinite plane geometry assumption
  }
  \item {Surface normals are not considered. That is why, cavities in
  geometry are problematic to render correctly without special treatment. And
  bump details are bluried}
  \item{Not that trivial importance sampling. See details in section
  \ref{section:burley_importance}}
\end{itemize}