\section{Scope of the topic}
\label{section:scope}

Recent achievements in the approximations of reflective and refractive materials with Bidirectional
Reflectance Distribution Functions and decoupling it from the light integration made a big step in
the physically based visualization \cite{pharr2010physically}. Together with a modern hardware, it
is now possible to visualize a wide range of scenarios required by everyday tasks in engineering,
marketing and visual effects domains. The absence of the a big number of ad-hoc parameters allows to
start rendering without labour intensive manual preparations needed in the past for certain
algorithms \cite{DBLP:conf/mam/KettnerRSJK15}.

But the rendering of the subsurface scattered materials remains a bottleneck for most of the light
transport simulation scenarios. In this paper I study Monte Carlo rendering of the subsurface
scattering materials, both in the context of light simulation and approximate rendering.
The main interest is in the correctness of the methods and the consistency of the results between
them. With the discussion of the optimisation techniques applicable in certain scenarios.

I focus on the methods of the rendering of highly scattering isotropic materials which usually
appear as optically thick objects. These are examples of the materials which, in most cases, meet
this condition:
\begin{itemize}
    \item Non-organic
    \begin{itemize}
        \item Plastics, rubbers or waxes
        \item Marble (polished, rough, dusty)
        \item Snow
        \item Non photo real (cartoon and artistic stylisation)
    \end{itemize}
    \item Organics
    \begin{itemize}
    \item Optically thick liquids (milk, some kinds of juice)
        \item Skin (sweaty, dry, makeup)
        \item Food (cheese, meat, bread, some kinds of fruits)
    \end{itemize}
\end{itemize}
These are examples of materials out of the scope:
\begin{itemize}
    \item Highly transparent liquids (dirty/dusty water, ocean water, wine)
    \item Fluids (air, fog, dust)
\end{itemize}
I consider only isotropic media which are modeled by means of geometric
polygonal meshes. I do not consider spectral scattering. Only monochromatic light is simulated.
There are papers describing the efficient process of combined rendering of RGB spectrum with
corresponding multiple importance sampling weights.

There were many publications in the recent years related to rendering of the Subsurface Scattering
using a concept of the Photon Beam Diffusion \cite{Habel:2013:PBD:2600890.2600896}. This technique
demonstrates significant results and robustness especially while in combination with other
sampling approaches as UPBP \emph{Unifying Points, Beams, and Paths in Volumetric Light Transport
Simulation} \cite{krivanek14upbp}.

Despite the high potential of the above mentioned approach, there are two main reasons why it is not
described in my work.

First, it is based on a Bidirectional Path Tracing
\cite{Lafortune:1996:RPM:275458.275468}. Even though it is a powerful algorithm solving many
complicated phenomenon, it usually requires very complex implementation and still not that
ubiquitous among the available production rendering engines. 

The second complication is the requirement of the additional precaching step as it is required for
classic photon mapping approach or photon mapping for volumetric rendering \cite{Jensen:1998:ESL:280814.280925}.
There are several disadvantages of using light caching for rendering:
\begin{itemize}
    \item{Undesired preprocessing step is unfriendly to progressive (iterative) rendering}
    \item{Density of the generated point cloud (or other structure) limits the scope of the
    available parameters. I.e. restricts the effective width of the SSS by minimal distance between
    points such that point density $\approx$ mean free path length}
    \item{Photon mapping introduce some extra control parameters which often require the end user to
    learn the details of the underlying algorithms}
    \item{Additional memory requirements}
    \item{Workflow complications for storing additional data}
    \item{Possible flickering artifacts during animation}
\end{itemize}
