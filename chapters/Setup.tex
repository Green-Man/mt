\chapter{Setup}
\label{chapter:Setup}

\section{Searchlight configuration}
Perpendicular light beam to infinite surface \cite{Jacques1995}

The term is: \gls{infiniteslab},
\comment{New comment Here. But where?}

\section{The Diffusion Approximation}
Applies in highly scattering media: relation to the Central Limit Theorem
The approach underlying the application of diffusion approximations is comparable to that underlying normal approximation for
sums of random variables. In the latter setting, the central limit theorem
permits one to replace the analytically intractable sum of random variables by
an appropriately chosen normal random variable. \cite{Donner:2006:TRI:1236937}

\section{Diffusion profiles}
\begin{itemize}
\item{Diffusion profiles can be measured from a real-world object
by analyzing scattering of laser or structured light patterns (Jensen et al. 2001, Tariq et al. 2006).}
\item{\cite{Nguyen:2007:GG:1407436} in \textit{GPU Gems 3} says that
the dipole and multipole functions mentioned already can be used to compute diffusion profiles
for any material with known scattering coefficients.}
\item{While knowing the correct diffusion profiles one can use fitting tool to approximate them \cite{Nguyen:2007:GG:1407436}}
\end{itemize}