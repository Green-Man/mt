\section{Christensen-Burley method}
The general problem discussed in this section is a search for a good diffusion approximation model
to fit Monte Carlo references by using analytic BSSRDF function. The most recent research in the topic
\cite{Burley:disney_siggraph15}, \cite{Christensen:2015:ARP:2775280.2792555} suggests the
\gls{BSSRDF} profile expressed as a combination of two exponential functions with corresponding
parametrization factor based on physical properties of the media and proper normalization to ensure
energy conservation. It was proposed as function which better fit Monte Carlo references than widely
used Gaussian and Cubic profiles \cite{King:2013:BIS:2504459.2504520}.

Short form of the Christensen-Burley diffuse reflectance profile:
\begin{equation}\label{eq:burley}
R_d(r) = As\dfrac{e^{-sr}+e^{-sr/3}}{8\pi r}
\end{equation}
Where:
\begin{itemize}
    \item{$r$ is a distance between the points where light enters and leaves the surface. Assumption
    of planar geometry applies}
    \item{ $A=\int_0^{\infty} R(r)2\pi rdr$ is the surface albedo (integral
    amount of the radiant exitance over whole area of the object relative to the
    overall irradiance). Surface albedo of the material depends on $\sigma_s$
    and $\sigma_a$. Section \ref{section:cb_parametrization}}
    \item{$s$ is a scaling factor function of $A$, empirically chosen by Christen and Burley
    to fit the Monte-Carlo references. The particular form of this function is discussed in section
    \ref{section:cb_modification} of this chapter below}
\end{itemize}

Note, that \emph{volume mean free path} $l = 1/(\sigma_s + \sigma_a)$ is included in the full
formulation of the Christensen-Burley profile:
\[
R_d(r) = As\frac{e^{-sr/l}+e^{-sr/3l}}{8\pi lr} = As\frac{e^{-sr^\prime}+e^{-sr^\prime/3}}{8\pi
r^\prime l^2}
\]
But the dependency on $l$ only affects the scale of the BSSRDF profile and not it's shape. The
integration of the of the additional spatial scaling does not show any principal difficulties. That
is why it can be included in the scaling factor and all further calculation will assume the
Christensen-Burley profile in the normalized form \ref{eq:burley}.

In contrast to the other diffusion approximation techniques proposed in the past (Dipole
(\cite{Jensen:2001:PMS:383259.383319}), Quantized diffusion
(\cite{D'Eon:2011:QMR:1964921.1964951})), this model already includes single scattering component,
so no expensive separate calculation is needed.

According to authors of the original paper, the main disadvantages of the model are:
\begin{itemize}
  \item {The model follows the infinite plane geometry assumption as all other Diffusion
  approximation models}
  \item {Surface normals are not considered. That is why, cavities of the
  geometry are problematic to render correctly without special treatment. Bump details are blurred}
  \item{Not that trivial importance sampling. See details in section
  \ref{section:burley_importance}}
\end{itemize}

Next section of this chapter contain the discussion of the Christensen-Burley BSSRDF in more
details.
