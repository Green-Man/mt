\section{Christensen-Burley profile importance sampling}
\label{section:burley_importance}

Importance sampling of the BSSRDF function is a vital part of the
rendering system used in practice. The short theory of importance sampling is given in the section
\ref{ImportanceSampling}. According to it, we need to derive \gls{CDF} of the profile and invert it to find
the corresponding sampling. Analogously to the derivation provided in the original paper
\cite{Christensen:2015:ARP:2775280.2792555}, the \gls{CDF} of the modified Christensen-Burley
\gls{BSSRDF} profile \ref{eq:burley_modified} is expressed by the following expression:
\[
\label{eq:burley_cdf}
CDF(R_d(r)) = \int_{0}^{r} R_d(x)2\pi xdx = \int_{0}^{r} A\frac{s e^{-sx}+t e^{-tx/3}}{8\pi x}2\pi
xdx = A(1-\frac{1}{4}e^{-sr}-\frac{3}{4}e^{-\frac{tr}{3}})
\]

Let $\xi$ be uniformly distributed random variable withing the range from 0 to 1. To find the
sampling of the radius $r$ according to the \ref{eq:burley_modified}, the following equation have to
be solved (inversion step):
\begin{equation}
\label{eq:inversion}
\xi=A(1-\frac{1}{4}e^{-sr}-\frac{3}{4}e^{-\frac{tr}{3}})
\end{equation}

This step would in theory result in the corresponding function $r(\xi)$ which can be used to
generate samples. But the fact that there is no analytical solution to equation
\ref{eq:inversion} makes it not that trivial to use in practice.
There are three main approaches to overcome this difficulty:
\begin{enumerate}
  \item Precompute the solution before rendering and use a lookup table with interpolation
  during the runtime
  \item Sample each of the exponential of the right side of the equation \ref{eq:burley_modified}
  separately and combine the result
  \item Solve the equation numerically in runtime for each sample on demand
\end{enumerate}

The derivative of CDF:
\[
\frac{dCDF(R_d(r))}{dr} = \frac{A}{4}(se^{-sr}+te^{-\frac{tr}{3}})
\]

Sampling each of the exponential separately leads to six importance Samples in
total in RGB mode.

\textbf{The main point of this section:}
Couple of Newton iterations to invert CFD show more robust results than
combination of the sampling of the each exponent separately.

\emph{Quote: Sampling can be difficult and expensive. Diffusion
integration requires distributing samples over the entire surface with known
density, preferably proportional to the diffusion profile. We can generate
such samples within a plane, but projecting this density through a solid may
lead to oversampling of some parts of the surface and undersampling of the
others.
Using MIS between multiple planes \cite{King:2013:BIS:2504459.2504520}  helps prevent
undersampling but oversampling is still a problem. Each plane sample also
requires ray probing to find surface hits, and if there are multiple
intersections we build a conditional CDF based on the distance to each point,
then stochastically choose a point to sample.}