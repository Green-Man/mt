% Included by MAIN.TEX

\clearemptydoublepage
\phantomsection
\addcontentsline{toc}{chapter}{Abstract}

\vspace*{2cm}
\begin{center}
{\Large \textbf{Abstract}}
\end{center}
\vspace{1cm}

Robust rendering of the wide range of subsurface scattering materials remains an unsolved problem in
computer graphics up to date. Light transport simulation in participating media is still one of the
most computationally intensive aspects of the objects visualisation for engineering or marketing
purposes and visual effects production. This fact makes it important to study both physically
correct methods for accurate materials representation and fast approximative solutions of the light
transport simulation.

This work discusses path tracing with Next Event Estimation approach for rendering of the dense
isotropic media without refractive boundaries and compare it to the classic unidirectional path
tracing.

Possible numerical issues caused by the large number of scattering events during path tracing with
large number of steps inside the media are described. The empirical solutions to overcome those
problems are given.

Another big part of the thesis is dedicated to the Diffusion Approximation with BSSRDF shading
model. We propose the improvement for Christensen-Burley precomputed profile and show empirical
parametrization approach to improve consistency with Monte Carlo references. Two variants of the
importance sampling of Christensen-Burley profile were implemented with comparison suggesting the
beneficial method.

The heuristic approach of defining the limitations of Diffusion approximation and relation to the
real world measured data is shown in the last chapter.
