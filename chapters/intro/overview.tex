%\section{Overview}
%\label{section:overview}
The context of the rendering application always dictate the constraints applied to the methods and
algorithms used by the software for the image generation.
Modern approaches to realistic rendering could be roughly divided into distinct groups:
physically based rendering (light simulation approach) and plausible rendering (approximate
rendering, phenomenological approach). I will use the terms light transport simulation and simply,
rendering, interchangeably in this work.

Physically based rendering tries to simulate interaction of the light and matter taking into account
the laws of physics and our theoretical knowledge about the nature.

Plausible rendering, on the other hand, is intended to mimic optical effects observed by the human
eye or captured by camera, accounting the audience perception.

Monte Carlo methods of the light simulation are usually perceived as physically based methods.
Sometimes they are also used for relatively fast way to achieve desired plausible results regardless
of the physical correctness.