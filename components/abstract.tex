% Included by MAIN.TEX

\clearemptydoublepage
\phantomsection
\addcontentsline{toc}{chapter}{Abstract}

\vspace*{2cm}
\begin{center}
{\Large \textbf{Abstract}}
\end{center}
\vspace{1cm}

Robust rendering of the wide range of subsurface scattering materials remains an unsolved problem in
computer graphics up to date. Light transport simulation in participating media is still one of the
most computationally intensive aspects of the objects visualisation for engineering or marketing
purposes and visual effects production. This fact makes it important to study both physically
correct methods for accurate materials representation and fast approximative solutions of the light
transport simulation.

This work discusses path tracing with Next Event Estimation approach for rendering of the dense
isotropic media without refractive boundaries and compare it to the classic unidirectional path
tracing.

The numerical issues caused by the large number of scattering events inside the media and the
proposed solutions are given. Another big part of the thesis is dedicated to the Diffusion
approximation with BSSRDF shading model. We propose the improvement for Christensen-Burley
precomputed profile and show empirical parametrization approach to improve consistency with Monte
Carlo references.

The heuristic approach of defining the limitations of Diffusion approximation and relation to the
real world measured data is shown in the last chapter.
