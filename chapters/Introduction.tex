\chapter{Introduction}
\label{chapter:Introduction}

\section{Intro}

\subsection{Real world meterial approach}
\begin{itemize}
    \item Organics
    \begin{itemize}
      \item Skin (sweaty, dry, makeup)
      \item Fruits (layered, uniform, non-uniform)
      \item Food (Cheese, meat, bread)
    \end{itemize}
    \item Non-organic
    \begin{itemize}
      \item Marble (polished, rough, dusty)
      \item Plastics (wax, etc othe non-organic uniform solid bodies)
    \end{itemize}
    \item Liquids (milk, wine, juice, dirty/dusty water, ocean water)
    \item Fluids (air, fog, dust)
    \item Snow
    \item Thin sheets (paper, foilage, cloth)
    \item Non photoreal (Cartoon style and stylisation)
\end{itemize}

\subsection{Things to consiger while choosing method}
\begin{description}
  \item [Lighting restrictions] \emph{Analytic light} sources and \emph{area lights and IBL} friendly
  \item [Multiple object friendly]
  \item [Preprocess] Is additional preprocess step needed (point cloud generation)
  \item [Parametrization] Artist friendly or physically correct
  \item [Performance] Is it a real time approximation for OpenGL
  \item [Flickering] How suitable for animation (does it introduce additional flickering)
  \item [Motion blur] How well it works with motion blur (SSS and volumetrics)
  \item [Closed objecs only] Does it supports single sided geometry
  \item Homogeneous or inhomogeneous
  \item Istropic or anisoisotropic media
\end{description}