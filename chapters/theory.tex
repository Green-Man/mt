\chapter{Theory}
\label{chapter:theory}


\section{Monte Carlo light integration in participating media}
\cite{Lafortune:1996:RPM:275458.275468} has short but decent description of the
integrals on the surface and in the media to integrate using with Monte Carlo
approach.
A nice picture explaining BDPT is also there.

\subsection{Russian roulette}
\label{subsection:rr}
Not a variance reduction in mathematical sense. Strictly speaking, it it even
increases the variance of the estimator. But reduces to cost of sampling in
average \cite{Veach:1998:RMC:927297}, \cite{Csi03variancereduction}

\ldots

\subsection{Emission}
\subsection{Absorption}
\subsection{Scattering}

\subsection{Henyey-Greenstein}
\cite{Gkioulekas:2013:IVR:2508363.2508377} made a great user study.
Take home notes from it: \gls{hg} is not general enough (can't represent some
materials). Uniform parametrization needs requires to expose $g^2$ to the user.

TODO: while inroducing phase function, try to render single photon inside the
media and visualize it with plotly as Andreas did in pseudo 3d. TO illustrate
random walk directions with different Henyey-Greenstein \textbf{g}.

Question: What are materials with high albedo and strong backward scattering
and high scattering coefficient? Do they exist? How they look like?

\section{The Diffusion Approximation}
Applies in highly scattering media: relation to the Central Limit Theorem
The approach underlying the application of diffusion approximations is comparable to that underlying normal approximation for
sums of random variables. In the latter setting, the central limit theorem
permits one to replace the analytically intractable sum of random variables by
an appropriately chosen normal random variable.

Probably first description is \cite{Stam1995}
Main paper: A Practical Model for Subsurface Light Transport \cite{Jensen:2001:PMS:383259.383319}

Geometric assumption on the surface as a flat, semi-infinite slab at (any
point). This is reasonable only if scattering distance is small with respect to
surface curvature and thickness. \textbf{Multiple scattering} diffusion
component is approximately determined. The \textbf{single scattering} is
separately integrated.

Following: A Rapid Hierarchical Rendering Technique for Translucent Materials\\
\cite{Jensen:2002:RHR:566570.566619} Introduces the idea of decoupling the calculations of surface
illumination and the sub-surface scattering effect in a two-pass method;
describes a fast hierarchical approach for evaluating subsurface scattering
and proposes a reparametrization of the BSSRDF parameters for easier user adjustment.

The techniques of Quantized diffusion \cite{D'Eon:2011:QMR:1964921.1964951} and
Improved diffusion \cite{Habel:2013:PBD:2600890.2600896} keeps the flat slab
assumption, but imploves multiple scattering approximation. The computation cost
is increased significantly.

\section{Donner's Empirical BSSRDF}
This strange \gls{BSSRDF}
\comment{I like the reasoning here}
Donner et al. in \cite{Donner:2009:EBM} claim that the sum of single scattering
simulation and diffusion approximation (isotropic multiple scattering) does not
able to capture materials with relatively large suspended particles (such as
orange juice or milk). Because of the significant amount of light in scattered
duirng to low-oreder scattering events and high absorbtion.
These materials exhibit significant absorption of the light as it propagates
through the material, so much of the energy is scattered back into
the environment near the point of incidence. This light exits in areas and
directions outside of the single scattering regime, but not in the high-order
multiple scattering regime.

In the same paper !(probably) there is a statement:
mean-free path gives the scale of the response for a given material. For two
materials that differ only by $l$, we can predict the response of the second
material by scaling the response of the first material by the ratio of their
mean-free-paths.

In this work they use Monte Carlo particle tracing to tabulate an empirical
BSSRDF model for a flat surface on a homogeneous semi-infinite volume.
They represented the hemispherical distribution of light leaving the surface,
depending on the angle of the incident light, the relative position of the
exitant light, and the physical parameters (volume albedo, mean free path
length, phase function, and index of refraction). Their tables took months to
compute and contain around 250MB of data.
\comment{Still can't find the data. But could be very helpful}

\section{About single scattering}
The iterative approach of finding refractive paths from inside the triangle
meshes is described here \cite{walter:inria-00388481}. Requires building a tree
and solving 2d Newton iteration for triangle meshes with shading normals.

The follow up on single scattering in participating media with refractive
bundries is \cite{holzschuch:hal-01083246} It is implemented in Mitsuba.

\section{Summary of the Improved Diffusion theory report}
Tech report from Habel et al. \cite{habel13cid} is similar to
it \cite{Habel:2013:PBD:2600890.2600896} TODO: What is the difference
From Radiance Transport Equation {insert it here} to the integral equation which
computes the outgoing radiance, Lo, at position, direction ($~x_o$,$~w_o$) and
BSSRDF \textit{insest one here}


\section{Quantized-diffusion model}
Original paper \cite{D'Eon:2011:QMR:1964921.1964951}
\begin{itemize}
    \item Improved diffusion theory
    \item Extended source term instead of just a dipole
    \item More realistic look with sharper features
    \item Much more complicated and time-consuming to calculate (DO I WANT TO COMPARE PERFORMANCE?)
\end{itemize}
D’Eon and Irving  greatly increased the accuracy of the dipole model, using a
better approximation for the diffusion inside the material. Their approach still
loses directional information, while \cite{holzschuch:hal-00760054} keep it. The
difference could be important for anisotropic materials.
The resulting diffuse reflectance profile was approximated as a sum of Gaussians.

\section{Double scattering paper}
\cite{Donner:2009:EBM} model is the most accurate to date (2013), but it has a
high memory cost (up to 250 MB for each material). It is also limited to a
single lobe. \cite{holzschuch:hal-00760054} extends and completes Donner's
study. They also provide a model that is less accurate, but much more compact
and can represent multiple lobes.

Still use that assumption of boundary as an infinite plane.

\section{Photon Beam diffusion}
The photon beam diffusion \cite{Habel:2013:PBD:2600890.2600896} had three main
contributions:
\begin{itemize}
    \item As accurate as quantized diffusion but much faster to evaluate,
    \item An accurate diffuse single-scattering model
    \item Elegant handling of oblique refraction into the material
\end{itemize}

\section{Diffusion profiles}
\comment{Think about how to organize this section}
\begin{itemize}
    \item{Diffusion profiles can be measured from a real-world object by
    analyzing scattering of laser \cite{Jensen:2001:PMS:383259.383319} or
    structured light patterns \cite{tariq_efficient_2006-1} Big variaty of
    liquids by dilusion was asquired in
    \cite{Narasimhan:2006:ASP:1141911.1141986} but only single scattering
    component (is it right)
    \item{And the most advanced method capturing both single and multple
    scattering is described \cite{Gkioulekas:2013:IVR:2508363.2508377}}}
    \item{\cite{Nguyen:2007:GG:1407436} in \textit{GPU Gems 3} says that the
    dipole and multipole functions mentioned already can be used to compute
    diffusion profiles for any material with known scattering coefficients.
    \item{In \cite{Frisvad:2014:DDM:2702692.2682629} introduces the directional
    dipole}. This improves the accuracy for e.g. non-perpendicular (oblique)
    illumination of objects with smooth refractive surfaces (such as milk, fruit
    juice, and other liquids with suspended particles) }
    \item{While knowing the correct diffusion profiles one can use fitting tool
    to approximate them \cite{Nguyen:2007:GG:1407436}}
\end{itemize}
