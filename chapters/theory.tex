\chapter{Theory}
\label{chapter:theory}


\section{Monte Carlo methods}
\cite{Lafortune:1996:RPM:275458.275468} has short but decent description of the
integrals on the surface and in the media to integrate using with Monte Carlo
approach.
A nice picture explaining BDPT is also there.

\subsection{Russian roulette}
\label{subsection:rr}
Not a variance reduction in mathematical sense. Strictly speaking, it it even
increases the variance of the estimator. But reduces to cost of sampling in
average \cite{Veach:1998:RMC:927297}, \cite{Csi03variancereduction}

\ldots

\section{Light simulation in participating media}
\subsection{Emission}
\subsection{Absorption}
\subsection{Scattering}

\subsection{Henyey-Greenstein}
\cite{Gkioulekas:2013:IVR:2508363.2508377} made a great user study.
Take home notes from it: \gls{hg} is not general enough (can't represent some
materials). Uniform parametrization needs requires to expose $g^2$ to the user.

TODO: while inroducing phase function, try to render single photon inside the
media and visualize it with plotly as Andreas did in pseudo 3d. TO illustrate
random walk directions with different Henyey-Greenstein \textbf{g}.

Question: What are materials with high albedo and strong backward scattering
and high scattering coefficient? Do they exist? How they look like?

\section{The Diffusion Approximation}
Applies to the highly scattering media: relation to the Central Limit Theorem
The approach underlying the application of diffusion approximations is comparable to that underlying normal approximation for
sums of random variables. In the latter setting, the central limit theorem
permits one to replace the analytically intractable sum of random variables by
an appropriately chosen normal random variable.

Probably first description is \cite{Stam1995}
Main paper: A Practical Model for Subsurface Light Transport \cite{Jensen:2001:PMS:383259.383319}

Geometric assumption on the surface as a flat, semi-infinite slab at (any
point). This is reasonable only if scattering distance is small with respect to
surface curvature and thickness. \textbf{Multiple scattering} diffusion
component is approximately determined. The \textbf{single scattering} is
separately integrated.
