\chapter{Introduction}
\label{chapter:Introduction}

\section{Intro}
\subsection{Real world meterial approach}
\begin{itemize}
    \item Organics
    \begin{itemize}
      \item Skin (sweaty, dry, makeup)
      \item Fruits (layered, uniform, non-uniform)
      \item Food (Cheese, meat, bread)
    \end{itemize}
    \item Non-organic
    \begin{itemize}
      \item Marble (polished, rough, dusty)
      \item Plastics (wax, etc othe non-organic uniform solid bodies)
    \end{itemize}
    \item Liquids (milk, wine, juice, dirty/dusty water, ocean water)
    \item Fluids (air, fog, dust)
    \item Snow
    \item Thin sheets (paper, foilage, cloth)
    \item Non photoreal (Cartoon style and stylisation)
\end{itemize}

\subsection{Things to consiger while choosing method}
\begin{description}
  \item [Lighting restrictions] \emph{Analytic light} sources and \emph{area lights and IBL} friendly
  \item [Multiple object friendly]
  \item [Preprocess] Is additional preprocess step needed (point cloud generation)
  \item [Parametrization] Artist friendly or physically correct
  \item [Performance] Is it a real time approximation for OpenGL
  \item [Flickering] How suitable for animation (does it introduce additional flickering)
  \item [Motion blur] How well it works with motion blur (SSS and volumetrics)
  \item [Closed objecs only] Does it supports single sided geometry
  \item Homogeneous or inhomogeneous
  \item Istropic or anisoisotropic media
\end{description}


\section{Skin}

Cubic, Gaussean (sum from GPU gems),

Cristensebn-Burley falloffs (implemented in newest Cycles)

\cite{EGWR:EGSR06:409-417} has formulas for the absorption and reduced scattering of the human skin. Also a lot of skin rendering results.

\section{Diffusion approximation}
Main paper: A Practical Model for Subsurface Light Transport

A Rapid Hierarchical Rendering Technique for Translucent Materials
Introduces the idea of decoupling the calculations of surface illumination and the sub-surface scattering effect in a two-pass method; describes a fast hierarchical approach for evaluating subsurface scattering and proposes a reparametrization of the BSSRDF parameters for easier user adjustment.

\section{Quantized-diffusion model \cite{D'Eon:2011:QMR:1964921.1964951}}
\begin{itemize}
\item Improved diffusion theory
\item Extended source term instead of just a dipole
\item More realistic look with sharper features
\item Much more complicated and time-consuming to calculate (DO I WANT TO COMPARE PERFORMANCE?)
\end{itemize}
The resulting diffuse reflectance profile was approximated as a sum of Gaussians.

\section{Photon Beam diffusion \cite{Habel:2013:PBD:2600890.2600896}}
The photon beam diffusion had three main contributions:
\begin{itemize}
\item As accurate as quantized diffusion but much faster to evaluate,
\item An accurate diffuse single-scattering model
\item Elegant handling of oblique refraction into the material
\end{itemize}

\section{Cristensebn-Burley method \cite{Christensen:2015:ARP:2775280.2792555}}
Diffusion models above require separate handling of single scattering, either by explicit ray tracing or separate integration.
Both methods are slow. Cristensebn-Burley model includes single scattering, so no
expensive separate calculation is needed.