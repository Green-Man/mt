\section{Volumetric path tracing with Next Event Estimation}
\label{section:udpt}
Although the classic Unidirectional Path Tracing \gls{UDPT} is proved to be a robust unbiased
estimator, it has some limitations.

A typical disadvantage of \gls{UDPT} is the high variance and subsequently strong noise component in
the rendered images of the scenes with highly non-uniform incoming illumination.
In practice, these scenes contain small but intensive emitters or direct sun light in the
image-based lighting scenes. The extreme worst case is the presence of the analytic light sources:
point lights with zero area or directional lights, having virtually infinite distance to the shaded
location.

Next Event Estimation (\gls{NEE}) is one of the common methods of reducing a variance of the path
tracing. The main idea of the \gls{NEE} is to estimate \textit{direct illumination} by sampling the
light source directly at each step of the path tracing. Then, the shadow test ray is needed to
ensure that the current light sampling position in visible by the surface.

\gls{NEE} can also be considered as the special case of the general Bidirectional Path Tracing
algorithm \gls{BDPT} \cite{Veach:94:BDPT}.
Using the vocabulary of the original paper, NEE algorithm is \textit{(m,2)-method} with the length
of light path equals 2 and length of the eye path $m\geq3$ is defined by the \gls{UDPT} settings.

The idea of direct light sampling can be applied to the volumetric path tracing as well.
The generalized theory of \gls{BDPT} in participating media was first described by Lafortune and
Willems in \cite{Lafortune:1996:RPM:275458.275468}.

Preforming the direct light estimation from inside the media with the refractive boundary proved to
be a non-trivial task. Due to the refraction on the boundary between materials with different index
of refraction (\gls{IOR}), the light from the emitter to the scattering point never travels by a
straight light.
Except the case of the normal incident rays. To satisfy the Fermat's principle \footnote{Fermat's
principle or the principle of least time states that the path taken between two points by a ray of
light is the path that can be traversed in the least time} we have to construct the path with the
middle point somewhere on the boundary.

There are no robust and fast enough methods for solving this complication in general way, to my
knowledge. Although, there are considerable iterative approaches described in the literature
last years \cite{holzschuch:hal-01083246}, \cite{10.1111:cgf.12681}, \cite{Koerner2016}.

In this work I have decided to consider the situation when the boundary refraction can be neglected.
In the other words, only the simplified case of materials with \gls{IOR}=1 is described.
This assumption leads to the approximation of the direct light path from inside the media to the
light source as a straight line. This certainly introduces an error in the simulation.
But the error is significant only for materials with low optical density. Light propagation in
highly scattering media, in contrast, is characterized by many scattering events. Which makes the
direction of the first ray less important for the future light simulation.

In practice, \gls{UDPT} almost always contains Russian Roulette optimization. Detailed description
is in section \ref{subsection:rr}. It proved to be a robust unbiased technique. But it may has some
non obvious numerical problems especially while being used in scenarios like volumetric rendering
(see chapter \ref{section:numerical})

