\chapter{Setup}
\label{chapter:Setup}

\section{Searchlight configuration}
Perpendicular light beam to infinite surface \cite{Jacques1995}

The term is: \gls{infiniteslab},

\section{Parametrization}
Artists prefer to define surface albedo and mean multiple scattering distance
(characteristic or visible depth of the SSS). The radiative transport scattering
coefficients must therefore be inferred by inverting the diffusion equation.

The scattering was parameterized by the volume scattering and absorption
coefficients $\sigma_s$ and $\sigma_a$ (or, equivalently, the volume
scattering albedo $\alpha = \sigma_s/\sigma_t = \sigma_s/(\sigma_s + \sigma_a)$
and volume mean free path length $l = 1/\sigma_t = 1/(\sigma_s + \sigma_a)$.)
In follow-up work, Jensen and Buhler \cite{Jensen:2002:RHR:566570.566619}
introduced a more intuitive parameterization of subsurface scattering:
surface albedo (diffuse surface reflectance) $A = \int\limits_0^\infty
R(r)2\pi r dr$ and diffuse mean free path length $l_d$ on the surface.

Such �albedo inversion� must typically done with iterative or tabulated methods,
both of which can have trouble with high albedos. Another challenge is that
diffusion profiles are often too blurry given that they typically only model
multiple scattering, leading to the use of two or more profiles combined in an
ad hoc manner.

\section{Diffusion profiles}
\comment{Think about how to organize this section}
\begin{itemize}
    \item{Diffusion profiles can be measured from a real-world object by
    analyzing scattering of laser \cite{Jensen:2001:PMS:383259.383319} or
    structured light patterns \cite{tariq_efficient_2006-1} Big variaty of
    liquids by dilusion was asquired in
    \cite{Narasimhan:2006:ASP:1141911.1141986} but only single scattering
    component (is it right)
    \item{And the most advanced method capturing both single and multple
    scattering is described \cite{Gkioulekas:2013:IVR:2508363.2508377}}}
    \item{\cite{Nguyen:2007:GG:1407436} in \textit{GPU Gems 3} says that the
    dipole and multipole functions mentioned already can be used to compute
    diffusion profiles for any material with known scattering coefficients.
    \item{In \cite{Frisvad:2014:DDM:2702692.2682629} introduces the directional
    dipole}. This improves the accuracy for e.g. non-perpendicular (oblique)
    illumination of objects with smooth refractive surfaces (such as milk, fruit
    juice, and other liquids with suspended particles) }
    \item{While knowing the correct diffusion profiles one can use fitting tool
    to approximate them \cite{Nguyen:2007:GG:1407436}}
\end{itemize}