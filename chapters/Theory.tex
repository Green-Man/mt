\chapter{Theory}
\label{chapter:Theory}

\section{Skin}

Cubic, Gaussean (sum from GPU gems),

Cristensebn-Burley falloffs 

\cite{EGWR:EGSR06:409-417} has formulas for the absorption and reduced scattering of the human skin. Also a lot of skin rendering results.

\section{The Diffusion Approximation}
Main paper: A Practical Model for Subsurface Light Transport \cite{Jensen:2001:PMS:383259.383319}

Following: A Rapid Hierarchical Rendering Technique for Translucent Materials\\
\cite{Jensen:2002:RHR:566570.566619} Introduces the idea of decoupling the calculations of surface
illumination and the sub-surface scattering effect in a two-pass method;
describes a fast hierarchical approach for evaluating subsurface scattering
and proposes a reparametrization of the BSSRDF parameters for easier user adjustment.

Applies in highly scattering media: relation to the Central Limit Theorem
The approach underlying the application of diffusion approximations is comparable to that underlying normal approximation for
sums of random variables. In the latter setting, the central limit theorem
permits one to replace the analytically intractable sum of random variables by
an appropriately chosen normal random variable.

Donner et al. in \cite{Donner:2009:EBM} claim that the sum of single scattering
simulation and diffusion approximation (isotropic multiple scattering) does not
able to capture such materials as orange juice. Because a significant amount of
light in scattered duirng to low-oreder scattering events and high absorbtion.
These materials exhibit significant absorption of the light as it propagates
through the material, so much of the energy is scattered back into
the environment near the point of incidence. This light exits in areas and
directions outside of the single scattering regime, but not in the high-order
multiple scattering regime.

In this work they use Monte Carlo particle tracing to tabulate an empirical
BSSRDF model for a flat surface on a homogeneous semi-infinite volume.
They represented the hemispherical distribution of light leaving the surface,
depending on the angle of the incident light, the relative position of the
exitant light, and the physical parameters (volume albedo, mean free path
length, phase function, and index of refraction). Their tables took months to
compute and contain around 250MB of data. (Still can't find thee data)

\section{Summary of \textbf{Classical and Improved Diffusion Theory for
Subsurface Scattering}}
Tech report from Habel et al. \cite{habel13cid} is similar to \cite{Habel:2013:PBD:2600890.2600896}
TODO: What is the difference
From Radiance Transport Equation {insert it here} to the integral equation which
computes the outgoing radiance, Lo, at position, direction (~ xo,~ wo) and
BSSRDF \textit{insest one here}


\section{Quantized-diffusion model}
\cite{D'Eon:2011:QMR:1964921.1964951}\\
\begin{itemize}
    \item Improved diffusion theory
    \item Extended source term instead of just a dipole
    \item More realistic look with sharper features
    \item Much more complicated and time-consuming to calculate (DO I WANT TO COMPARE PERFORMANCE?)
\end{itemize}
The resulting diffuse reflectance profile was approximated as a sum of Gaussians.

\section{Photon Beam diffusion}
The photon beam diffusion \cite{Habel:2013:PBD:2600890.2600896} had three main
contributions:
\begin{itemize}
    \item As accurate as quantized diffusion but much faster to evaluate,
    \item An accurate diffuse single-scattering model
    \item Elegant handling of oblique refraction into the material
\end{itemize}

\section{Cristensebn-Burley method}
(implemented in newest Cycles) \cite{Christensen:2015:ARP:2775280.2792555}\\
Diffusion models above require separate handling of single scattering, either by explicit ray tracing or separate integration.
Both methods are slow. Cristensebn-Burley model includes single scattering, so no
expensive separate calculation is needed.